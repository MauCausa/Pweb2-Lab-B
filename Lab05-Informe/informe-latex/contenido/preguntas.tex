\section{Pregunta: Por cada integrante del equipo, resalte un aprendizaje que adquiri o al momento de estudiar
Django. No se reprima de ser detallista. Coloque su nombre entre parentesis para saber que es
su aporte.}
\begin{itemize}
	\item Mauricio Congona Manrique: Al momento de trabajar con Django, resalto el patrón MVC (Modelo Vista Controlador) porque ayuda a organizar mejor el código separando la lógica del código del interfaz de usuario; además del estilo modular.
        \item Sebastian Alfonso Huacasi: Un aprendizaje importante que adquirí al estudiar Django es la importancia de seguir buenas prácticas de desarrollo web. Django es un framework poderoso y flexible que permite crear aplicaciones web de manera rápida y eficiente.
        \item Maxs Forocca Mamani: En el laboratorio 05 de Django, se estudió este framework y su funcionamiento, utilizando plantillas (Templates) que acceden a la información de los modelos enviada por las vistas (Views). Esta información se puede visualizar a través de URLs asignadas.
        \item Klismann Chancuaña Alvis: Trabajar con Django nos ofrece una experiencia de desarrollo web eficiente y estructurada, con una amplia gama de características y una comunidad sólida. Es ideal para construir aplicaciones web complejas y de alto rendimiento, y su enfoque en la seguridad y la reutilización del código lo convierte en una opción confiable para proyectos a largo plazo.
\end{itemize}
